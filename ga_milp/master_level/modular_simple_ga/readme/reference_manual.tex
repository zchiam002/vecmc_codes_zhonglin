\documentclass[a4paper]{article}

\usepackage[margin=1in]{geometry}
\usepackage{amsmath}
\usepackage{amssymb}
\usepackage{graphicx}

\begin{document}
\title{Mono-Objective Genetic Algorithm in Python}
\author {}
\maketitle

\centerline{Reference Manual} 
\newpage

\tableofcontents

\clearpage

\section{Introduction}

This is an implementation of the Genetic Algorithm in Python. 
\\
\\The added features for faster resolution times are as follows:

\begin{enumerate}
\item Handling variable types differently (continuous, binary, discrete)
\item Conversion of variables into binary equivalent 
\item User defined choice of selection mechanism 
\item Parallel computing 
\end{enumerate}

\subsection{Basic genetic algorithm structure}

The basic structure of the genetic algorithm is as follows:

\begin{enumerate}
\item Initialize a population of agents
\item Evaluate the fitness of the agents in the population 
\item For the predefined number of generations: 
\begin{enumerate}
\item Select parents for the creation of offsprings
\item Use the parents to create offsprings 
\item Apply mutation mechanism for diversity 
\item Evaluate the fitness of this new population 
\end{enumerate}
\item Extract the agents with best fitness 
\end{enumerate}

\section{Installation}

At the moment, this code only works in Windows\textregistered. The folder has to be copied into the directory 'C:\textbackslash'  as some internal commands are hard-coded.

\subsection{Required python packages}

\begin{enumerate}
\item pandas 
\item numpy 
\item datetime 
\item matplotlib
\item math 
\item copy
\item random
\item multiprocessing 
\end{enumerate}

\subsection{Using the package}
Important files and locations to take note of:

\begin{enumerate}
\item Input data: \texttt{ ga\_mono\_simple\-setup.py}
\begin{enumerate}
\item Population size (e.g. population = 200).
\item Number of generations to run the algorithm for (e.g. generations = 1000)
\item Selection mechanism to determine the agents populating the parent pool
\begin{enumerate}
\item Available options: 1. \texttt{roulette\_wheel} and 2. \texttt{tournament\_selection}
\item 
\end{enumerate}
\item Determine the size of the parent pool 
\begin{enumerate}
\item The variable  \texttt{crossover\_perc} is used to determine the size of the parent pool
\item e.g. \texttt{crossover\_perc} = 0.5, means that the parent pool will be 50\% of the population size
\end{enumerate}
\end{enumerate}
\end{enumerate}





\section{Modules}
\section{References}

\end {document}
